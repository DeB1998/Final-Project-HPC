\documentclass{article}
\usepackage{amsmath}
\usepackage{amssymb}
\usepackage{multicol}
\usepackage{geometry}
\usepackage{float}
\usepackage[unicode=true, psdextra]{hyperref}
\usepackage{caption}
\usepackage{subcaption}
\usepackage{graphicx}
\usepackage{expl3}
\usepackage{xparse}
\usepackage{array}
\usepackage{booktabs}
\usepackage{tabularx}
\usepackage[table, svgnames]{xcolor}
\usepackage{csquotes}
\usepackage{titling}
\usepackage{enumitem}
\usepackage{comment}
\usepackage{lipsum}
\usepackage{bm}
\usepackage{mathtools}
\usepackage{titlesec}
\usepackage{tikz}
\usepackage{color}

% https://tex.stackexchange.com/questions/88929/vertical-table-lines-are-discontinuous-with-booktabs

\setlength{\droptitle}{-2.3em} %-5em
\geometry{left=1cm,right=1cm,top=1.7cm,bottom=2cm}
\graphicspath{{images/}}
\captionsetup[table]{skip=5pt}
\titlespacing*{\section}{0pt}{5pt}{5pt}
\titlespacing*{\subsection}{0pt}{5pt}{5pt}
\titlespacing*{\subsubsection}{0pt}{5pt}{5pt}

\hypersetup{
    pdftitle={Parallelization of the SLINK clustering algorithm},
    pdfauthor={Alessio De Biasi (870288); Jonathan Gobbo (870506)},
    hidelinks,
    pdfcreator={XeLaTeX}}

\title{Parallelization of the SLINK clustering algorithm\vspace{-1.25ex}}
\author{Alessio De Biasi (870288) \and Jonathan Gobbo (870506)}
\date{August 2022}

% Options for packages loaded elsewhere
\PassOptionsToPackage{unicode}{hyperref}
\PassOptionsToPackage{hyphens}{url}

\newenvironment{tablelayout}[1][]{
    \begin{table}[H]
        \centering
        \caption{#1}
        \begin{tabular}[H]{ccccccc}
            \hline
            & s1 & s2 & s3 & s4 & tot & \\
            \hline
            }{
            \\
            \hline
        \end{tabular}
    \end{table}
}

\newenvironment{tablelayout2}{
    \begin{table}[H]
        \centering
        \begin{tabular}[H]{lrrrrrr}
            \hline
            DS & \centering s1 & \centering s2 & \centering s3 & s4 & s5 & tot \\
            \hline
            }{
            \\
            \hline
        \end{tabular}
    \end{table}
}

\ExplSyntaxOn
\NewDocumentCommand{\formatNumber}{mm} {
\seq_set_split:Nnn \values {.} {#1}
\seq_item:Nn \values {1}
\int_compare:nNnTF{\seq_count:N \values} = {2}
    {.\footnotesize    \seq_item:Nn \values {2}}
    {}
\normalsize$\,$#2
}
\NewDocumentEnvironment{tableLayout}{mb}{
\setlength\intextsep{2mm}
\renewcommand{\divisor}{\midrule}
\seq_set_split:Nnn \l_rows_seq { \\ } {#2}
\begin{table}[H]
\centering
\begin{tabularx}{0.98\linewidth}[H]{
>{\raggedright}p{1.8em}|
%!{\color{Gainsboro!50!Lavender}\vline width 0.75em}
>{\raggedleft}X
>{\raggedleft}X
>{\raggedleft}X
>{\raggedleft}X
>{\raggedleft\arraybackslash}X
}
\toprule
\textbf{DS} &
\multicolumn{1}{c}{\hspace{1.2em}\textbf{S1}} &
\multicolumn{1}{c}{\textbf{S2}} &
\multicolumn{1}{c}{\textbf{S3}} &
\multicolumn{1}{c}{\textbf{S4}} &
\multicolumn{1}{c}{\textbf{Total}} \\
\toprule
\seq_use:Nn \l_rows_seq {\\}\\
\bottomrule
\end{tabularx}
\caption{#1}
\end{table}
}{}

\NewDocumentEnvironment{tableLayout2}{mb}{
\setlength\intextsep{2mm}
\renewcommand{\divisor}{\midrule}
\seq_set_split:Nnn \l_rows_seq { \\ } {#2}
\begin{table}[h]
\centering
\begin{tabularx}{0.98\linewidth}[H]{
>{\raggedright}p{1.8em}|
%!{\color{Gainsboro!50!Lavender}\vline width 0.75em}
>{\raggedleft}p{1.8em}
>{\raggedleft}X
>{\raggedleft}X
>{\raggedleft}X
>{\raggedleft}X
>{\raggedleft\arraybackslash}p{3.2em}
}
\toprule
\textbf{DS} &
\multicolumn{1}{c}{\hspace{0.4em}\textbf{S1}} &
\multicolumn{1}{c}{\textbf{S2}} &
\multicolumn{1}{c}{\textbf{S3}} &
\multicolumn{1}{c}{\textbf{S4}} &
\multicolumn{1}{c}{\textbf{S5}} &
\multicolumn{1}{c}{\textbf{Total}} \\
\toprule
\seq_use:Nn \l_rows_seq {\\}\\
\bottomrule
\end{tabularx}
\caption{#1}
\end{table}
}{\vspace{-4pt}}
\ExplSyntaxOff

\newcommand{\divisor}{& \\[-2.25ex]\hline& \\[-2.25ex]}
\newcommand{\s}{$\,$s}
\newcommand{\ms}{$\,$ms}
\newcommand{\m}{$\,$m$\ $}

\setlength{\itemsep}{0pt}
\setlength{\parskip}{0pt}
\setlist{itemsep=0.25mm,
% Margin of the whole list
topsep=0.25mm,
% Padding between each element of the list
parsep=0.25mm}

\begin{document}
\twocolumn
\maketitle

\section{Introduction}

The purpose of this project is to implement the SLINK clustering algorithm and to apply
parallelization techniques to improve its performance.

All the implementations of the clustering algorithms are written in \texttt{C++20} and they are
compiled with the GCC G++ 12.1.1 (Red Hat) compiler, using the \texttt{-O3} and
\texttt{-march=native} flags.

In particular, we tested the clustering algorithm implementations on two datasets:
\begin{itemize}
\item Accelerometer\footnote{\url{https://archive.ics.uci.edu/ml/datasets/Accelerometer}} (A),
composed
of 153'000 samples, each of which with 5 attributes. For the tests we have performed, we have
selected only the last 3 of them;
\item Generated (G), composed of 100'000 samples, each of which with 45 randomly generated real
attributes uniformly distributed in the range 0-100.
\end{itemize}

We will report the mean execution time of 3 executions of each implementation on the two datasets,
run on a machine equipped with an AMD Ryzen 7 1700 CPU with 8 cores and 16 threads, 16 GB of RAM,
and running Fedora Workstation 36.

In particular, we measured the time taken by each step of the pseudo-code (reported as S1, S2, S3
and S4) to be executed.

In all the tests we used two \texttt{std::vector} data structures to hold the values of $\pi$ and
$\lambda$.

Note that the reported total times include also some extra time, which is spent by the various
implementations to perform operations like allocating and de-allocating the memory, or copying
the iterators.

Moreover, for space reasons, in some tests we do not report the results obtained using 2 and 16
threads.

\hypertarget{implementations}{
\section{Implementations}
\label{implementations}}

\hypertarget{sequential}{
\subsection{Sequential}
\label{sequential}}

We first implemented the sequential version of the clustering algorithm.
This implementation is just the translation in \texttt{C++} code of the pseudo-code reported in
the slides.

The data is supplied as a
\texttt{std::vector\textless{}double\ *\textgreater{}}, i.e., a vector
of pointers to heap-allocated arrays, where each of these arrays holds the attributes of one data
sample.

\begin{tableLayout}{Execution times of the sequential implementation}
A & 19\ms & 1\m 30\s & \formatNumber{37.330}{s} & \formatNumber{20.271}{s} & 2\m 28\s \\
\divisor
G & 19\ms & 8\m 59\s & \formatNumber{16.246}{s} & \formatNumber{12.802}{s} & 9\m 28\s
\end{tableLayout}


\hypertarget{parallel-distance}{%
\subsection{Multithreaded distance computation}\label{parallel-distance}}

From the previous tests results, we saw that the computation of the distances (i.e., the step
\textit{2} of the pseudo-code) is the most expensive step of the algorithm.

Since each distance can be computed independently from the others, their computation can be
easily made parallel by using the \emph{OpenMP} library.

The supplied data has the same memory layout as before.

\begin{tableLayout}{Execution times of the first parallel implementation}
A2 & 10\ms & \formatNumber{43.043}{s} & \formatNumber{39.568}{s} & \formatNumber{23.286}{s} & 1\m
46\s \\
A4 & 11\ms & \formatNumber{29.641}{s} & \formatNumber{36.144}{s} & \formatNumber{23.609}{s} & 1\m
29\s \\
A8 & 11\ms & \formatNumber{20.940}{s} & \formatNumber{35.260}{s} & \formatNumber{23.541}{s} & 1\m
20\s \\
A12 & 10\ms & \formatNumber{24.402}{s} & \formatNumber{36.713}{s} & \formatNumber{24.204}{s} &
1\m 25s \\
A16 & 11\ms & \formatNumber{27.201}{s} & \formatNumber{47.172}{s} & \formatNumber{26.257}{s} &
1\m 41\s \\
\divisor
G2 & 13\ms & 4\m 12\s & \formatNumber{15.228}{s} & \formatNumber{13.855}{s} & 4\m 41\s \\
G4 & 15\ms & 2\m 23\s & \formatNumber{15.555}{s} & \formatNumber{14.881}{s} & 2\m 53\s \\
G8 & 14\ms & 1\m 34\s & \formatNumber{15.305}{s} & \formatNumber{14.977}{s} & 2\m 04\s \\
G12 & 12\ms & 1\m 43\s & \formatNumber{17.569}{s} & \formatNumber{15.308}{s} & 2\m 16\s \\
G16 & 12\ms & 1\m 42\s & \formatNumber{20.592}{s} & \formatNumber{15.668}{s} & 2\m 18\s
\end{tableLayout}

\hypertarget{parallel-sse-avx}{%
\subsection{SSE and AVX in the computation of the distances}\label{parallel-sse-avx}}

Since the attributes of each data sample to cluster are stored contiguously in memory, we decided
to improve the previous solution using the SSE and AVX instructions when computing the distances.

Indeed, to compute the distance between two points, the algorithm perform the same instructions
(difference and square) for every pair of attributes.

\begin{tableLayout}{Execution times of the implementation using SSE}
A2 & 11\ms & \formatNumber{46.922}{s} & \formatNumber{32.362}{s} & \formatNumber{21.116}{s} & 1\m
40\s \\
A4 & 11\ms & \formatNumber{34.329}{s} & \formatNumber{35.250}{s} & \formatNumber{23.791}{s} & 1\m
33\s \\
A8 & 9\ms & \formatNumber{27.252}{s} & \formatNumber{35.007}{s} & \formatNumber{23.633}{s} & 1\m
26\s \\
A12 & 10\ms & \formatNumber{27.328}{s} & \formatNumber{37.252}{s} & \formatNumber{24.346}{s} &
1\m 29\s \\
A16 & 12\ms & \formatNumber{28.596}{s} & \formatNumber{47.285}{s} & \formatNumber{26.313}{s} &
1\m 42\s \\
\divisor
G2 & 14\ms & 4\m 29\s & \formatNumber{16.732}{s} & \formatNumber{14.733}{s} & 5\m 00\s \\
G4 & 15\ms & 2\m 33\s & \formatNumber{19.175}{s} & \formatNumber{16.085}{s} & 3\m 08\s \\
G8 & 13\ms & 1\m 39\s & \formatNumber{19.319}{s} & \formatNumber{15.894}{s} & 2\m 15\s \\
G12 & 13\ms & 1\m 43\s & \formatNumber{18.766}{s} & \formatNumber{15.457}{s} & 2\m 18\s \\
G16 & 14\ms & 1\m 39\s & \formatNumber{21.918}{s} & \formatNumber{15.986}{s} & 2\m 17\s
\end{tableLayout}

\begin{tableLayout}{Execution times of the implementation using AVX}
A2 & 8\ms & \formatNumber{39.106}{s} & \formatNumber{38.321}{s} & \formatNumber{23.127}{s} & 1\m
41\s \\
A4 & 10\ms & \formatNumber{24.656}{s} & \formatNumber{37.981}{s} & \formatNumber{23.659}{s} & 1\m
26\s \\
A8 & 8\ms & \formatNumber{17.814}{s} & \formatNumber{37.826}{s} & \formatNumber{23.772}{s} & 1\m
19\s \\
A12 & 8\ms & \formatNumber{18.416}{s} & \formatNumber{40.067}{s} & \formatNumber{24.198}{s} & 1\m
23\s \\
A16 & 10\ms & \formatNumber{19.824}{s} & \formatNumber{52.452}{s} & \formatNumber{26.367}{s} &
1\m 39\s \\
\divisor
G2 & 13\ms & 4\m 34\s & \formatNumber{14.860}{s} & \formatNumber{14.236}{s} & 5\m 03\s \\
G4 & 15\ms & 2\m 37\s & \formatNumber{15.111}{s} & \formatNumber{15.457}{s} & 3\m 07\s \\
G8 & 14\ms & 1\m 42\s & \formatNumber{15.143}{s} & \formatNumber{15.660}{s} & 2\m 13\s  \\
G12 & 14\ms & 1\m 49\s & \formatNumber{17.418}{s} & \formatNumber{15.516}{s} & 2\m 22\s \\
G16 & 14\ms & 1\m 44\s & \formatNumber{20.112}{s} & \formatNumber{15.970}{s} & 2\m 20\s
\end{tableLayout}

We expected this implementation to work faster than the previous one but, as the results show,
this implementation generally works slightly slower than the previous one.

Investigating using the \textit{perf} tool, we discovered that most of the time the CPU is
stalled due to a high rate of cache misses (about 80\% of all the data memory accesses).

\hypertarget{a-new-data-structure}{%
\subsection{A new data structure}\label{a-new-data-structure}}

Therefore, we decided to change the data structure holding the data samples to cluster in flavour
to a more a contiguous one.

In particular, we chose to organize the data as a unique heap-allocated array, where the data
samples, as well as the attributes of each of them, are stored one after the other.

%, like the attributes of each sample.

\begin{tableLayout}{Execution times of the implementation using a contiguous data structure and
SSE (2 and 16 threads tests omitted)}
A4 & 4\ms & \formatNumber{11.770}{s} & \formatNumber{34.453}{s} & \formatNumber{23.509}{s} & 1\m
10\s \\
A8 & 4\ms & \formatNumber{7.531}{s} & \formatNumber{35.604}{s} & \formatNumber{23.674}{s} & 1\m
07\s \\
A12 & 4\ms & \formatNumber{8.277}{s} & \formatNumber{37.068}{s} & \formatNumber{24.300}{s} & 1\m
10\s \\
\divisor
G4 & 9\ms & \formatNumber{44.391}{s} & \formatNumber{14.319}{s} & \formatNumber{15.880}{s} & 1\m
15\s \\
G8 & 8\ms & \formatNumber{37.005}{s} & \formatNumber{14.162}{s} & \formatNumber{15.937}{s} & 1\m
07\s \\
G12 & 8\ms & \formatNumber{37.132}{s} & \formatNumber{15.271}{s} & \formatNumber{15.902}{s} & 1\m
08\s
\end{tableLayout}
\vspace{-5pt}
\begin{tableLayout}{Execution times of the implementation using a contiguous data structure and
AVX(2 and 16 threads tests omitted)}
A4 & 4\ms & \formatNumber{12.462}{s} & \formatNumber{42.709}{s} & \formatNumber{23.633}{s} & 1\m
18\s \\
A8 & 4\ms & \formatNumber{8.315}{s} & \formatNumber{42.649}{s} & \formatNumber{23.517}{s} & 1\m
14\s \\
A12 & 4\ms & \formatNumber{8.368}{s} & \formatNumber{42.744}{s} & \formatNumber{23.809}{s} & 1\m
15\s \\
\divisor
G4 & 10\ms & \formatNumber{46.830}{s} & \formatNumber{14.452}{s} & \formatNumber{15.770}{s} & 1\m
17\s \\
G8 & 8\ms & \formatNumber{38.992}{s} & \formatNumber{14.454}{s} & \formatNumber{16.222}{s} & 1\m
10\s \\
G12 & 8\ms & \formatNumber{39.331}{s} & \formatNumber{15.580}{s} & \formatNumber{15.969}{s} & 1\m
11\s
\end{tableLayout}

As we can see from these results, making the data samples contiguous in memory improves
significantly the performance of the algorithm.

Indeed, when a thread loads one data sample, it is very likely that with the same memory
transaction more data samples are brought to the cache, hence, when the thread accesses the
following data samples, they are already in cache.

\hypertarget{stage-3-parallel}{
\subsection{Make the insertion of a new point parallel}
\label{stage-3-parallel}}

After having made parallel the computation of the distances, we focused on parallelizing the
next most expensive step of the algorithm, which is the insertion of a new point to
the dendrogram (i.e., the step \textit{3} of the pseudo-code).

However, there is no chance to make it parallel because, when the element at index $i$ is
analyzed, the algorithm requires that all the elements before $i$ have already been analyzed.

Therefore, if we give to each thread an element to analyze in a round-robin fashion, then there will
be always a thread working while all the other threads will be blocked waiting for the active
thread to complete its work.

\hypertarget{stage-4-parallel}{
\subsection{Make the rearrangement of the structure of the dendrogram parallel}
\label{stage-4-parallel}}

Therefore, we decided to try to parallelize the third most expensive step of the algorithm, which
is the rearrangement of the structure of the
dendrogram after a new point is added (i.e., the step \textit{4} of the pseudo-code).

Since each iteration of the \texttt{for} loop is independent from the others, making it parallel
it is almost trivial if we use the \emph{OpenMP} library.

Like the last implementation, data is supplied as a unique heap-allocated array.

% TODO:
For space reasons, we omitted the results of the implementation when using SSE.
%, since its execution times are slightly different than the ones of AVX.


%For space reasons, we omitted the results of the implementation when using SSE instructions, since
%it takes slightly more time to be executed with respect to the AVX version.
%%%%%%%%%%%




\begin{tableLayout}{Execution times of the implementation executing in parallel the computation
of the distances (using threads and AVX instructions) as well as the rearrangement of the
structure of the dendrogram (2 and 16 threads tests omitted)}
A4 & 4\ms & \formatNumber{10.599}{s} & \formatNumber{42.538}{s} & \formatNumber{8.939}{s} & 1\m
02\s \\
A8 & 4\ms & \formatNumber{5.836}{s} & \formatNumber{43.255}{s} & \formatNumber{5.659}{s} &
\formatNumber{54.760}{s} \\
A12 & 4\ms & \formatNumber{6.895}{s} & \formatNumber{45.326}{s} & \formatNumber{4.489}{s} &
\formatNumber{56.720}{s} \\
\divisor
G4 & 9\ms & \formatNumber{46.341}{s} & \formatNumber{15.961}{s} & \formatNumber{6.888}{s} & 1\m
09\s \\
G8 & 8\ms & \formatNumber{38.337}{s} & \formatNumber{16.110}{s} & \formatNumber{3.792}{s} &
\formatNumber{58.254}{s} \\
G12 & 8\ms & \formatNumber{39.039}{s} & \formatNumber{17.603}{s} & \formatNumber{3.408}{s} & 1\m
00\s
\end{tableLayout}

\vspace{-10pt}

\hypertarget{failed-attempt}{
\subsection{Devising a new algorithm}
\label{failed-attempt}}

The last most expensive step of the algorithm is the initialization $\pi$ and $\lambda$ (i.e.,
the step \textit{1} of the pseudo-code). However, as the results show, even if we make it
parallel, we will gain only few milliseconds.

Therefore, to improve further the performance of the algorithm, we tried devising a new one. This
new algorithm builds incrementally the dendrogram like the pseudo-code does, but in a different way.

In particular, when the algorithm adds to the dendrogram a new data sample $d$, it first computes
the closest sample $c$, and their distance $l$. Then, it traverses the path on the dendrogram
from the leaf representing $c$ to the root.

Indeed, since $c$ is the closest sample to $d$, then $d$ must be linked to one of the clusters
containing $c$, and such a cluster can be found only on that path. In particular, this cluster
is represented by the last edge $\varepsilon$ in the path with height smaller than $l$. The
algorithm just needs to add a new edge linking $\varepsilon$ to $d$, and then it links the
new edge with the parent of $\varepsilon$.

Then, the algorithm continues traversing the path so to possibly update the height of the
remaining edges, because adding $d$ to a cluster may have made it closer to the cluster it
joins to.

Representing the dendrogram as a binary tree using a linked structure, each edge of the path to
traverse is just a pair of nodes.
Since the dendrogram is not rearranged, each of these pairs can be analyzed independently.
This implies that several threads may analyze such pairs in parallel requiring almost no
synchronization. One of these threads will add the new edge, while all the others will possibly
update the heights of the edges.

However, there are some cases where the dendrogram is actually rearranged. Indeed, suppose that
the cluster $\{a, b\}$ joins the cluster $\{e, f\}$. If the data sample
to add $d$ is closer to $b$, then it will be added to the first cluster, and this may make the
data sample $e$ closer to the cluster $\{a, b, d\}$ more than to $f$.

Thus, adding the data sample $d$ to the dendrogram may require to rearrange some of the edges.
This implies that the edges in the path to traverse may change while the threads are traversing it.
To avoid inconsistencies, we require much more synchronization between the threads, which may force
them to work sequentially, hence reducing the performance.

Therefore, we abandoned this algorithm, and we tried to implement some micro-optimizations so to
improve the performance the algorithm we have used so far.

\hypertarget{micro-optimization-partial-sum}{%
\subsection{Partial sums micro-optimization}\label{micro-optimization-partial-sum}}

To compute the distance between two data samples, the previous implementations using SSE and AVX
instructions first load from memory 2 and 4 attributes respectively. Then, they compute the
square of the pairwise attributes difference, and they accumulate the result on a stack-allocated
variable, which requires to horizontally sum all the values in the extended register.
All these four operations are repeated until no more attributes are left.

The first micro-optimization we have implemented accumulates the partial sums in an
extended register instead of in a
stack-allocated variable, in order to perform the horizontal sum only once at the end of the
computation, and to avoid writing to memory each partial result.

The new implementation combines this micro-optimization with all the parallelization strategies
we have implemented so far. The data is still supplied as a unique heap-allocated array.

Also in this case, for space reasons, we omitted the results of the implementation when using SSE.

\begin{tableLayout}{Execution times of the first micro-optimized implementation using AVX (2 and
16 threads tests omitted)}
A4 & 4\ms & \formatNumber{10.174}{s} & \formatNumber{41.862}{s} & \formatNumber{8.887}{s} & 1\m
01\s \\
A8 & 4\ms & \formatNumber{5.820}{s} & \formatNumber{42.804}{s} & \formatNumber{5.592}{s} &
\formatNumber{54.227}{s} \\
A12 & 4\ms & \formatNumber{6.868}{s} & \formatNumber{45.697}{s} & \formatNumber{4.526}{s} &
\formatNumber{57.102}{s} \\
\divisor
G4 & 11\ms & \formatNumber{31.854}{s} & \formatNumber{17.229}{s} & \formatNumber{5.854}{s} & 1\m
05\s \\
G8 & 10\ms & \formatNumber{37.064}{s} & \formatNumber{16.846}{s} & \formatNumber{3.590}{s} &
\formatNumber{57.515}{s} \\
G12 & 9\ms & \formatNumber{37.005}{s} & \formatNumber{17.897}{s} & \formatNumber{3.414}{s} &
\formatNumber{58.330}{s}
\end{tableLayout}

\hypertarget{micro-optimization-no-square-root}{%
\subsection{Square roots micro-optimization}\label{micro-optimization-no-square-root}}

The last micro-optimization we have implemented extends the previous one by using the
square of the distances instead of the distances themselves. The computation of the square roots
is performed only at end of the execution of the algorithm and only on the computed values of
$\lambda$.

Indeed, ignoring possible rounding errors, this solution works because the following property
holds for
non-negative values (as the distances are):

\[
\sqrt{n^2} \geq \sqrt{m^2} \quad \iff \quad n^2 \geq m^2
\]

\begin{tableLayout2}{Execution times of the second micro-optimized implementation using SSE}
A2 & 4\ms & \formatNumber{10.961}{s} & \formatNumber{37.824}{s} & \formatNumber{12.933}{s} &
190$\,\mu$s & 1\m 02\s\\
A4 & 4\ms & \formatNumber{6.641}{s} & \formatNumber{41.826}{s} & \formatNumber{8.989}{s} & 112$\,
\mu$s & \formatNumber{57.466}{s}\\
A8 & 4\ms & \formatNumber{3.903}{s} & \formatNumber{42.693}{s} & \formatNumber{5.592}{s} & 58$\,
\mu$s & \formatNumber{52.199}{s}\\
A12 & 5\ms & \formatNumber{4.603}{s} & \formatNumber{45.701}{s} & \formatNumber{4.549}{s} & 69$\,
\mu$s & \formatNumber{54.864}{s}\\
A16 & 5\ms & \formatNumber{4.067}{s} & \formatNumber{51.38}{s} & \formatNumber{4.222}{s} & 52$\,
\mu$s & \formatNumber{59.340}{s}\\
\divisor
G2 & 9\ms & 1\m 5\s & \formatNumber{15.592}{s} & \formatNumber{9.995}{s} & 113\,$\mu$s & 1\m 30\s\\
G4 & 9\ms & \formatNumber{43.109}{s} & \formatNumber{16.769}{s} & \formatNumber{6.865}{s} & 66$\,
\mu$s & 1\m 07\s\\
G8 & 9\ms & \formatNumber{36.077}{s} & \formatNumber{16.618}{s} & \formatNumber{3.834}{s} & 35$\,
\mu$s & \formatNumber{56.544}{s}\\
G12 & 9\ms & \formatNumber{36.747}{s} & \formatNumber{17.697}{s} & \formatNumber{3.430}{s} &
48$\,\mu$s & \formatNumber{57.890}{s}\\
G16 & 5\ms & \formatNumber{36.383}{s} & \formatNumber{20.554}{s} & \formatNumber{3.100}{s} &
38$\,\mu$s & 1\m 01\s
\end{tableLayout2}
\vspace{-4pt}
\begin{tableLayout2}{Execution times of the second micro-optimized implementation using AVX}
A2 & 4\ms & \formatNumber{11.439}{s} & \formatNumber{37.881}{s} & \formatNumber{13.146}{s} &
216$\,\mu$s & 1\m 02\s\\
A4 & 4\ms & \formatNumber{7.106}{s} & \formatNumber{42.571}{s} & \formatNumber{8.888}{s} & 115$\,
\mu$s & \formatNumber{58.576}{s}\\
A8 & 4\ms & \formatNumber{4.147}{s} & \formatNumber{42.281}{s} & \formatNumber{5.607}{s} & 61$\,
\mu$s & \formatNumber{52.046}{s}\\
A12 & 5\ms & \formatNumber{5.271}{s} & \formatNumber{44.500}{s} & \formatNumber{4.595}{s} & 71$\,
\mu$s & \formatNumber{54.376}{s}\\
A16 & 5\ms & \formatNumber{4.386}{s} & \formatNumber{50.834}{s} & \formatNumber{4.346}{s} & 52$\,
\mu$s & \formatNumber{59.578}{s}\\
\divisor
G2 & 9\ms & \formatNumber{53.895}{s} & \formatNumber{16.303}{s} & \formatNumber{10.436}{s} &
124$\,\mu$s & 1\m 21\s\\
G4 & 10\ms & \formatNumber{41.312}{s} & \formatNumber{16.961}{s} & \formatNumber{6.453}{s} &
67$\,\mu$s & 1\m 05\s\\
G8 & 8\ms & \formatNumber{37.104}{s} & \formatNumber{16.829}{s} & \formatNumber{3.715}{s} & 35$\,
\mu$s & \formatNumber{57.663}{s}\\
G12 & 8\ms & \formatNumber{37.015}{s} & \formatNumber{17.697}{s} & \formatNumber{3.430}{s} &
47$\,\mu$s & \formatNumber{58.427}{s} \\
G16 & 9\ms & \formatNumber{37.781}{s} & \formatNumber{20.467}{s} & \formatNumber{3.212}{s} &
38$\,\mu$s & 1\m 01\s
\end{tableLayout2}

In the previous tables, S5 reports the time taken to compute the square roots only.

\hypertarget{sequential-linearized}{
\subsection{A new data structure for the sequential implementation}
\label{sequential-linearized}}

To compute the speed-ups of all the previous parallel implementations, we need to also execute
the sequential implementation with data supplied as a unique heap-allocated array.


%\loremTableSequential
\begin{tableLayout}{Execution times of the sequential implementation using a contiguous data
structure to store the data samples}
A & 4\ms & \formatNumber{27.978}{s} & \formatNumber{28.764}{s} & \formatNumber{20.663}{s} & 1\m
17\s \\
\divisor
G & 16\ms & 2\m 24\s & \formatNumber{16.307}{s} & \formatNumber{13.705}{s} & 2\m 54\s
\end{tableLayout}

As we can see, this memory layout drastically improves the performance of the sequential
implementation, even without using any parallelization technique.

Indeed, storing all the data samples contiguously in memory greatly improves the use of the cache,
as we observed using the \textit{perf} tool.

\hypertarget{Data layouts comparison}{
\subsection{Data layouts comparison}
\label{data-layout-comparison}}

Finally, we report the results of the tests we have performed on the last parallel implementation
of the clustering algorithm, but supplying the data as a \texttt{std::vector<double *>} (i.e., a
vector of heap allocated arrays each holding the attributes of one data sample) like in the first
parallel implementation.

% TODO: S5 reports
In the following tables, S5 reports the time taken to compute the square roots only.
%%%%

%\loremTable

\begin{tableLayout2}{Execution times of the second micro-optimized implementation using SSE and
\texttt{std::vector<double*>}}
A2 & 9\ms & \formatNumber{42.708}{s} & \formatNumber{41.007}{s} & \formatNumber{17.475}{s} &
170$\,\mu$s & 1\m 41\s \\
A4 & 6\ms & \formatNumber{34.021}{s} & \formatNumber{42.132}{s} & \formatNumber{9.814}{s} & 97$\,
\mu$s & 1\m 26\s \\
A8 & 6\ms & \formatNumber{29.379}{s} & \formatNumber{42.149}{s} & \formatNumber{6.243}{s} & 51$\,
\mu$s & 1\m 18\s \\
A12 & 6\ms & \formatNumber{29.705}{s} & \formatNumber{43.549}{s} & \formatNumber{5.491}{s} &
72$\,\mu$s & 1\m 18\s \\
A16 & 7\ms & \formatNumber{30.903}{s} & \formatNumber{48.340}{s} & \formatNumber{5.206}{s} &
54$\,\mu$s & 1\m 24\s \\
\divisor
G2 & 9\ms & 2\m 52\s & \formatNumber{17.576}{s} & \formatNumber{10.547}{s} & 118$\,\mu$s & 3\m
20\s \\
G4 & 9\ms & 1\m 47\s & \formatNumber{19.144}{s} & \formatNumber{6.986}{s} & 66$\,\mu$s & 2\m 13\s \\
G8 & 9\ms & 1\m 24\s & \formatNumber{18.963}{s} & \formatNumber{4.404}{s} & 35$\,\mu$s & 1\m 48\s \\
G12 & 10\ms & 1\m 31\s & \formatNumber{20.374}{s} & \formatNumber{4.026}{s} & 49$\,\mu$s & 1\m
56\s \\
G16 & 10\ms & 1\m 31\s & \formatNumber{22.532}{s} & \formatNumber{3.765}{s} & 39$\,\mu$s & 1\m 57\s
\end{tableLayout2}

\begin{tableLayout2}{Execution times of the second micro-optimized implementation using AVX and
\texttt{std::vector<double*>}}
A2 & 7\ms & \formatNumber{33.418}{s} & \formatNumber{39.571}{s} & \formatNumber{15.797}{s} &
181$\,\mu$s & 1\m 29\s \\
A4 & 7\ms & \formatNumber{20.032}{s} & \formatNumber{40.853}{s} & \formatNumber{9.820}{s} & 95$\,
\mu$s & 1\m 11\s \\
A8 & 7\ms & \formatNumber{13.387}{s} & \formatNumber{42.220}{s} & \formatNumber{5.761}{s} & 61$\,
\mu$s & 1\m 01\s \\
A12 & 8\ms & \formatNumber{16.225}{s} & \formatNumber{44.615}{s} & \formatNumber{4.966}{s} &
68$\,\mu$s & 1\m 05\s \\
A16 & 9\ms & \formatNumber{19.212}{s} & \formatNumber{50.916}{s} & \formatNumber{4.934}{s} &
52$\,\mu$s & 1\m 15\s \\
\divisor
G2 & 14\ms & 2\m 27\s & \formatNumber{16.368}{s} & \formatNumber{10.641}{s} & 126$\,\mu$s & 2\m
54\s\\
G4 & 13\ms & 1\m 35\s & \formatNumber{17.435}{s} & \formatNumber{7.007}{s} & 69$\,\mu$s & 1\m
59\s \\
G8 & 13\ms & 1\m 22\s & \formatNumber{17.152}{s} & \formatNumber{4.432}{s} & 35$\,\mu$s & 1\m
44\s \\
G12 & 13\ms & 1\m 30\s & \formatNumber{20.023}{s} & \formatNumber{4.109}{s} & 53$\,\mu$s & 1\m
54\s \\
G16 & 14\ms & 1\m 29\s & \formatNumber{21.576}{s} & \formatNumber{3.801}{s} & 42$\,\mu$s & 1\m 55\s
\end{tableLayout2}

\hypertarget{speed-up}{
\subsection{Speed-up}
\label{speed-up}}

%2.1 - 2.11

When storing the data samples in a \texttt{std::vector<double*>}, we reached a speed-up of 2.43
for the
\textit{Accelerometer} dataset and 5.46 for the \textit{Generated} one, both executing
the AVX parallel implementation described in \ref{data-layout-comparison} with 8 threads.

Instead, when storing the data samples in a unique array, we obtained a speed-up of 1.48 for the
\textit{Accelerometer} dataset exeuting the AVX parallel implementation described in
\ref{micro-optimization-no-square-root} with 8 threads, while we obtained a speed-up of 3.08 for
the \textit{Generated} dataset executing the SSE parallel implementation described in
\ref{micro-optimization-no-square-root} with 8 threads.

The key takeaway is that the parallel techniques we have applied achieve better speed-ups when
the data samples are composed of many attributes, like for the \textit{Generated} dataset.

% TODO:
Moreover, we get the highest speed-ups when using 8 threads, since we are exploiting all the
available CPUs.
%which is the maximum parallel degree we can achieve on the CPU we have used.
%%%%

In conclusion, among all the scheduling policies offered by the \textit{OpenMP} library, we have
found that the \texttt{static} one is the most effective to parallelize the \texttt{for} loops.
\vspace{-6pt}

% Maybe: Are the various implementations effective in improving the performance


\newlength{\graphShift}
\newlength{\axesLineWidth}
\newlength{\gridLineWidth}
\newlength{\graphLineWidth}
\setlength{\graphShift}{0mm}
\setlength{\axesLineWidth}{2pt}
\setlength{\gridLineWidth}{1pt}
\setlength{\graphLineWidth}{2pt}

\definecolor{gridColor}{HTML}{CCCCCC}
\definecolor{labelColor}{HTML}{000000}
\definecolor{parallel1Color}{HTML}{587DE5}
\definecolor{parallel2Color}{HTML}{F79B15}
\definecolor{parallel3Color}{HTML}{E8513E}
\definecolor{parallel4Color}{HTML}{72B127}
\definecolor{parallel5Color}{HTML}{D36B06}
\definecolor{parallel6Color}{HTML}{9F47B0}
\definecolor{parallel7Color}{HTML}{7566D0}
\definecolor{parallel8Color}{HTML}{F07629}
\definecolor{parallel9Color}{HTML}{00B3B3}
\definecolor{parallel10Color}{HTML}{17A1DA}
\definecolor{parallel11Color}{HTML}{B9B800}

% 1 = 87
% 5 = 15
% 4 units = 72 mm

% 1 = 87
% 5 = 15
% 2 units = 72 mm


\begin{figure}[H]
\begin{minipage}{\linewidth}
\centering
\begin{subfigure}[t]{.47\linewidth}%
\begin{tikzpicture}[scale=0.47]
    %\draw (0mm, 0mm) rectangle (100mm, 130mm);
    % Horizontal grid
    % Total 72 for 1.5 --> 48 for 1 ---> 24 for 0.5
    %\draw[gridColor,line width=\gridLineWidth] (10mm, 96mm) -- (95mm, 96mm);
    \draw[gridColor,line width=\gridLineWidth, yshift=\graphShift] (95mm, 88.2mm) -- (10mm, 88.2mm)
    node[labelColor, anchor=east] {2.4};
    \draw[gridColor,line width=\gridLineWidth, yshift=\graphShift] (95mm, 74.6mm) -- (10mm, 74.6mm)
    node[labelColor, anchor=east] {2.2};
    \draw[gridColor,line width=\gridLineWidth, yshift=\graphShift] (95mm, 61mm) -- (10mm, 61mm)
    node[labelColor, anchor=east] {2};
    \draw[gridColor,line width=\gridLineWidth, yshift=\graphShift] (95mm, 47.4mm) -- (10mm, 47.4mm)
    node[labelColor, anchor=east] {1.8};
    \draw[gridColor,line width=\gridLineWidth, yshift=\graphShift] (95mm, 33.8mm) -- (10mm, 33.8mm)
    node[labelColor, anchor=east] {1.6};
    \draw[gridColor,line width=\gridLineWidth, yshift=\graphShift] (95mm, 20.2mm) -- (10mm, 20.2mm)
    node[labelColor, anchor=east] {1.4};
    % Vertical grid
    % (85-68)/2 = 8.5 mm of margin left and right
    \draw[gridColor,line width=\gridLineWidth, yshift=\graphShift] (18.5mm, 95mm) -- (18.5mm, 10mm)
    node[labelColor,
    anchor=north] {2};
    \draw[gridColor,line width=\gridLineWidth, yshift=\graphShift] (35.5mm, 95mm) -- (35.5mm, 10mm)
    node[labelColor,
    anchor=north] {4};
    \draw[gridColor,line width=\gridLineWidth, yshift=\graphShift] (52.5mm, 95mm) -- (52.5mm, 10mm)
    node[labelColor,
    anchor=north] {8};
    \draw[gridColor,line width=\gridLineWidth, yshift=\graphShift] (69.5mm, 95mm) -- (69.5mm, 10mm)
    node[labelColor,
    anchor=north] {12};
    \draw[gridColor,line width=\gridLineWidth, yshift=\graphShift] (86.5mm, 95mm) -- (86.5mm, 10mm)
    node[labelColor,
    anchor=north] {16};
    % Lines
    \draw[parallel1Color, line width=\graphLineWidth, yshift=\graphShift]
    (18.5mm, 19.94mm) --
    (35.5mm, 38.08mm) --
    (52.5mm, 50.80mm) --
    (69.5mm, 43.40mm) --
    (86.5mm, 24.64mm);
    
    \draw[parallel2Color, line width=\graphLineWidth, yshift=\graphShift]
    (18.5mm, 25.64mm) --
    (35.5mm, 33.22mm) --
    (52.5mm, 42.02mm) --
    (69.5mm, 38.08mm) --
    (86.5mm, 23.67mm);
    
    \draw[parallel3Color, line width=\graphLineWidth, yshift=\graphShift]
    (18.5mm, 24.64mm) --
    (35.5mm, 42.02mm) --
    (52.5mm, 52.39mm) --
    (69.5mm, 46.25mm) --
    (86.5mm, 26.66mm);
    
    \draw[parallel10Color, line width=\graphLineWidth, yshift=\graphShift]
    (18.5mm, 24.64mm) --
    (35.5mm, 42.02mm) --
    (52.5mm, 54.03mm) --
    (69.5mm, 54.03mm) --
    (86.5mm, 44.81mm);
    
    \draw[parallel11Color, line width=\graphLineWidth, yshift=\graphShift]
    (18.5mm, 38.08mm) --
    (35.5mm, 66.75mm) --
    (52.5mm, 89.98mm) --
    (69.5mm, 79.83mm) --
    (86.5mm, 59.19mm);
    % Axes
    \draw[line width=\axesLineWidth, yshift=\graphShift] (95mm, 10mm) -- (10mm, 10mm) -- (10mm,
    95mm);
    \fill[line width=\axesLineWidth, yshift=\graphShift] (7.5mm, 92mm) -- (10mm, 97mm) -- (12.5mm,
    92mm) --
    cycle;
    % Labels
    \draw[yshift=\graphShift] (4mm, 100mm) node[labelColor, anchor=north] {$S$};
    \draw[yshift=\graphShift] (96mm, 9mm) node[labelColor, anchor=north] {$p$};
\end{tikzpicture}
%
\setlength{\abovecaptionskip}{0pt}%
\caption*{\textit{Accelerometer} dataset with data layout \texttt{std::vector<double *>}}%
\end{subfigure}%
%
\hspace{5mm}%
%
\begin{subfigure}[t]{.47\linewidth}%
\begin{tikzpicture}[scale=0.47]
    %\draw (0mm, 0mm) rectangle (100mm, 100mm);
    % Horizontal grid
    % Total 81 for 4.5 --> 18 for 1
    %\draw[gridColor,line width=\gridLineWidth] (10mm, 96mm) -- (95mm, 96mm);
    \draw[gridColor,line width=\gridLineWidth, yshift=\graphShift] (95mm, 82.25mm) -- (10mm, 82
    .25mm)
    node[labelColor, anchor=east] {5};
    \draw[gridColor,line width=\gridLineWidth, yshift=\graphShift] (95mm, 61mm) -- (10mm, 61mm)
    node[labelColor, anchor=east] {4};
    \draw[gridColor,line width=\gridLineWidth, yshift=\graphShift] (95mm, 39.75mm) -- (10mm, 39
    .75mm) node[labelColor, anchor=east] {3};
    \draw[gridColor,line width=\gridLineWidth, yshift=\graphShift] (95mm, 18.5mm) -- (10mm, 18.5mm)
    node[labelColor, anchor=east] {2};
    % Vertical grid
    % (85-68)/2 = 8.5 mm of margin left and right
    \draw[gridColor,line width=\gridLineWidth, yshift=\graphShift] (18.5mm, 95mm) -- (18.5mm, 10mm)
    node[labelColor, anchor=north] {2};
    \draw[gridColor,line width=\gridLineWidth, yshift=\graphShift] (35.5mm, 95mm) -- (35.5mm, 10mm)
    node[labelColor, anchor=north] {4};
    \draw[gridColor,line width=\gridLineWidth, yshift=\graphShift] (52.5mm, 95mm) -- (52.5mm, 10mm)
    node[labelColor, anchor=north] {8};
    \draw[gridColor,line width=\gridLineWidth, yshift=\graphShift] (69.5mm, 95mm) -- (69.5mm, 10mm)
    node[labelColor, anchor=north] {12};
    \draw[gridColor,line width=\gridLineWidth, yshift=\graphShift] (86.5mm, 95mm) -- (86.5mm, 10mm)
    node[labelColor, anchor=north] {16};
    % Lines
    \draw[parallel1Color, line width=\graphLineWidth, yshift=\graphShift]
    (18.5mm, 18.95mm) --
    (35.5mm, 45.77mm) --
    (52.5mm, 73.34mm) --
    (69.5mm, 64.75mm) --
    (86.5mm, 63.46mm);
    
    \draw[parallel2Color, line width=\graphLineWidth, yshift=\graphShift]
    (18.5mm, 16.23mm) --
    (35.5mm, 40.20mm) --
    (52.5mm, 65.41mm) --
    (69.5mm, 63.46mm) --
    (86.5mm, 64.10mm);
    
    \draw[parallel3Color, line width=\graphLineWidth, yshift=\graphShift]
    (18.5mm, 15.83mm) --
    (35.5mm, 40.55mm) --
    (52.5mm, 66.75mm) --
    (69.5mm, 61.00mm) --
    (86.5mm, 62.21mm);
    
    \draw[parallel10Color, line width=\graphLineWidth, yshift=\graphShift]
    (18.5mm, 36.35mm) --
    (35.5mm, 66.75mm) --
    (52.5mm, 87.76mm) --
    (69.5mm, 80.05mm) --
    (86.5mm, 79.16mm);
    
    \draw[parallel11Color, line width=\graphLineWidth, yshift=\graphShift]
    (18.5mm, 45.37mm) --
    (35.5mm, 77.43mm) --
    (52.5mm, 92.06mm) --
    (69.5mm, 81.88mm) --
    (86.5mm, 80.96mm);
    % Axes
    \draw[line width=\axesLineWidth, yshift=\graphShift] (95mm, 10mm) -- (10mm, 10mm) -- (10mm,
    95mm);
    \fill[line width=\axesLineWidth, yshift=\graphShift] (7.5mm, 92mm) -- (10mm, 97mm) -- (12.5mm,
    92mm) --
    cycle;
    % Labels
    \draw[yshift=\graphShift] (4mm, 100mm) node[labelColor, anchor=north] {$S$};
    \draw[yshift=\graphShift] (96mm, 9mm) node[labelColor, anchor=north] {$p$};

\end{tikzpicture}%
%
\setlength{\abovecaptionskip}{0pt}%
\caption*{\textit{Generated} dataset with data layout \texttt{std::vector<double *>}}%
\end{subfigure}%
\end{minipage}
%
%
\begin{minipage}{\linewidth}
\centering
\begin{subfigure}[t]{.47\linewidth}%
\begin{tikzpicture}[scale=0.47]
    %\draw (0mm, 0mm) rectangle (100mm, 130mm);
    % Horizontal grid
    % Total 72 for 1.5 --> 48 for 1 ---> 24 for 0.5
    %\draw[gridColor,line width=\gridLineWidth] (10mm, 96mm) -- (95mm, 96mm);
    \draw[gridColor,line width=\gridLineWidth, yshift=\graphShift] (95mm, 79.06mm) -- (10mm, 79
    .06mm) node[labelColor, anchor=east] {1.45};
    \draw[gridColor,line width=\gridLineWidth, yshift=\graphShift] (95mm, 63.13mm) -- (10mm, 63
    .13mm) node[labelColor, anchor=east] {1.3};
    \draw[gridColor,line width=\gridLineWidth, yshift=\graphShift] (95mm, 47.19mm) -- (10mm, 47
    .19mm) node[labelColor, anchor=east] {1.15};
    \draw[gridColor,line width=\gridLineWidth, yshift=\graphShift] (95mm, 31.25mm) -- (10mm, 31
    .25mm) node[labelColor, anchor=east] {1};
    \draw[gridColor,line width=\gridLineWidth, yshift=\graphShift] (95mm, 15.31mm) -- (10mm, 15
    .31mm) node[labelColor, anchor=east] {0.85};
    % Vertical grid
    % (85-68)/2 = 8.5 mm of margin left and right
    \draw[gridColor,line width=\gridLineWidth, yshift=\graphShift] (18.5mm, 95mm) -- (18.5mm, 10mm)
    node[labelColor,
    anchor=north] {2};
    \draw[gridColor,line width=\gridLineWidth, yshift=\graphShift] (35.5mm, 95mm) -- (35.5mm, 10mm)
    node[labelColor, anchor=north] {4};
    \draw[gridColor,line width=\gridLineWidth, yshift=\graphShift] (52.5mm, 95mm) -- (52.5mm, 10mm)
    node[labelColor, anchor=north] {8};
    \draw[gridColor,line width=\gridLineWidth, yshift=\graphShift] (69.5mm, 95mm) -- (69.5mm, 10mm)
    node[labelColor, anchor=north] {12};
    \draw[gridColor,line width=\gridLineWidth, yshift=\graphShift] (86.5mm, 95mm) -- (86.5mm, 10mm)
    node[labelColor, anchor=north] {16};
    % Lines
    \draw[parallel4Color, line width=\graphLineWidth, yshift=\graphShift]
    (18.5mm, 29.89mm) --
    (35.5mm, 41.88mm) --
    (52.5mm, 47.11mm) --
    (69.5mm, 41.88mm) --
    (86.5mm, 28.56mm);
    
    \draw[parallel5Color, line width=\graphLineWidth, yshift=\graphShift]
    (18.5mm, 23.57mm) --
    (35.5mm, 29.89mm) --
    (52.5mm, 35.56mm) --
    (69.5mm, 34.08mm) --
    (86.5mm, 27.27mm);
    
    \draw[parallel6Color, line width=\graphLineWidth, yshift=\graphShift]
    (18.5mm, 35.56mm) --
    (35.5mm, 56.96mm) --
    (52.5mm, 76.50mm) --
    (69.5mm, 71.09mm) --
    (86.5mm, 56.96mm);
    
    \draw[parallel7Color, line width=\graphLineWidth, yshift=\graphShift]
    (18.5mm, 40.23mm) --
    (35.5mm, 59.12mm) --
    (52.5mm, 76.50mm) --
    (69.5mm, 68.53mm) --
    (86.5mm, 54.86mm);
    
    \draw[parallel8Color, line width=\graphLineWidth, yshift=\graphShift]
    (18.5mm, 56.96mm) --
    (35.5mm, 68.53mm) --
    (52.5mm, 82.33mm) --
    (69.5mm, 76.50mm) --
    (86.5mm, 63.67mm);
    
    \draw[parallel9Color, line width=\graphLineWidth, yshift=\graphShift]
    (18.5mm, 56.96mm) --
    (35.5mm, 66.06mm) --
    (52.5mm, 82.33mm) --
    (69.5mm, 76.50mm) --
    (86.5mm, 63.67mm);
    % Axes
    \draw[line width=\axesLineWidth, yshift=\graphShift] (95mm, 10mm) -- (10mm, 10mm) -- (10mm,
    95mm);
    \fill[line width=\axesLineWidth, yshift=\graphShift] (7.5mm, 92mm) -- (10mm, 97mm) -- (12.5mm,
    92mm) --
    cycle;
    % Labels
    \draw[yshift=\graphShift] (4mm, 100mm) node[labelColor, anchor=north] {$S$};
    \draw[yshift=\graphShift] (96mm, 9mm) node[labelColor, anchor=north] {$p$};
\end{tikzpicture}
%
\setlength{\abovecaptionskip}{0pt}%
\caption*{\textit{Accelerometer} dataset with unique array data layout}%
\end{subfigure}%
%
\hspace{5mm}%
%
\begin{subfigure}[t]{.47\linewidth}%
\begin{tikzpicture}[scale=0.47]
    %\draw (0mm, 0mm) rectangle (100mm, 130mm);
    % Horizontal grid
    % Total 81 for 4.5 --> 18 for 1
    %\draw[gridColor,line width=\gridLineWidth] (10mm, 96mm) -- (95mm, 96mm);
    \draw[gridColor,line width=\gridLineWidth, yshift=\graphShift] (95mm, 80mm) -- (10mm, 80mm)
    node[labelColor, anchor=east] {2.9};
    \draw[gridColor,line width=\gridLineWidth, yshift=\graphShift] (95mm, 65mm) -- (10mm, 65mm)
    node[labelColor, anchor=east] {2.6};
    \draw[gridColor,line width=\gridLineWidth, yshift=\graphShift] (95mm, 50mm) -- (10mm, 50mm)
    node[labelColor, anchor=east] {2.3};
    \draw[gridColor,line width=\gridLineWidth, yshift=\graphShift] (95mm, 35mm) -- (10mm, 35mm)
    node[labelColor, anchor=east] {2};
    \draw[gridColor,line width=\gridLineWidth, yshift=\graphShift] (95mm, 20mm) -- (10mm, 20mm)
    node[labelColor, anchor=east] {1.7};
    % Vertical grid
    % (85-68)/2 = 8.5 mm of margin left and right
    \draw[gridColor,line width=\gridLineWidth, yshift=\graphShift] (18.5mm, 95mm) -- (18.5mm, 10mm)
    node[labelColor, anchor=north] {2};
    \draw[gridColor,line width=\gridLineWidth, yshift=\graphShift] (35.5mm, 95mm) -- (35.5mm, 10mm)
    node[labelColor, anchor=north] {4};
    \draw[gridColor,line width=\gridLineWidth, yshift=\graphShift] (52.5mm, 95mm) -- (52.5mm, 10mm)
    node[labelColor, anchor=north] {8};
    \draw[gridColor,line width=\gridLineWidth, yshift=\graphShift] (69.5mm, 95mm) -- (69.5mm, 10mm)
    node[labelColor, anchor=north] {12};
    \draw[gridColor,line width=\gridLineWidth, yshift=\graphShift] (86.5mm, 95mm) -- (86.5mm, 10mm)
    node[labelColor, anchor=north] {16};
    % Lines
    \draw[parallel4Color, line width=\graphLineWidth, yshift=\graphShift]
    (18.5mm, 22.00mm) --
    (35.5mm, 51.00mm) --
    (52.5mm, 64.85mm) --
    (69.5mm, 62.94mm) --
    (86.5mm, 61.09mm);
    
    \draw[parallel5Color, line width=\graphLineWidth, yshift=\graphShift]
    (18.5mm, 22.88mm) --
    (35.5mm, 47.99mm) --
    (52.5mm, 59.29mm) --
    (69.5mm, 57.54mm) --
    (86.5mm, 51.00mm);
    
    \draw[parallel6Color, line width=\graphLineWidth, yshift=\graphShift]
    (18.5mm, 24.69mm) --
    (35.5mm, 61.09mm) --
    (52.5mm, 85.00mm) --
    (69.5mm, 80.00mm) --
    (86.5mm, 73.10mm);
    
    \draw[parallel7Color, line width=\graphLineWidth, yshift=\graphShift]
    (18.5mm, 38.57mm) --
    (35.5mm, 68.85mm) --
    (52.5mm, 87.63mm) --
    (69.5mm, 85.00mm) --
    (86.5mm, 77.62mm);
    
    \draw[parallel8Color, line width=\graphLineWidth, yshift=\graphShift]
    (18.5mm, 31.67mm) --
    (35.5mm, 64.85mm) --
    (52.5mm, 90.36mm) --
    (69.5mm, 87.63mm) --
    (86.5mm, 77.62mm);
    
    \draw[parallel9Color, line width=\graphLineWidth, yshift=\graphShift]
    (18.5mm, 42.41mm) --
    (35.5mm, 68.85mm) --
    (52.5mm, 87.63mm) --
    (69.5mm, 85.00mm) --
    (86.5mm, 77.62mm);
    % Axes
    \draw[line width=\axesLineWidth, yshift=\graphShift] (95mm, 10mm) -- (10mm, 10mm) -- (10mm,
    95mm);
    \fill[line width=\axesLineWidth, yshift=\graphShift] (7.5mm, 92mm) -- (10mm, 97mm) -- (12.5mm,
    92mm) --
    cycle;
    % Labels
    \draw[yshift=\graphShift] (4mm, 100mm) node[labelColor, anchor=north] {$S$};
    \draw[yshift=\graphShift] (96mm, 9mm) node[labelColor, anchor=north] {$p$};

\end{tikzpicture}%
%
\setlength{\abovecaptionskip}{0pt}%
\caption*{\textit{Generated} dataset with unique array data layout}%
\end{subfigure}%
\end{minipage}
%
%
% Legend
\begin{subfigure}{.47\linewidth}%
\hspace*{6pt}
\newlength{\marginLeft}%
\setlength{\marginLeft}{20pt}%
\begin{tikzpicture}
    % Legend
    \fill[parallel1Color] (4mm, 28mm) rectangle (7mm, 25mm) node[labelColor,
    anchor=mid west, yshift=0.25\baselineskip, xshift=-2pt] {2.2};
    \fill[parallel2Color] (15mm, 28mm) rectangle (18mm, 25mm)
    node[labelColor, anchor=mid west, yshift=0.25\baselineskip, xshift=-2pt] {2.3 \footnotesize
    SSE};
    \fill[parallel3Color] (33mm, 28mm) rectangle (36mm, 25mm)
    node[labelColor, anchor=mid west, yshift=0.25\baselineskip, xshift=-2pt] {2.3 \footnotesize
    AVX};
    \fill[parallel4Color] (51mm, 28mm) rectangle (54mm, 25mm)
    node[labelColor, anchor=mid west, yshift=0.25\baselineskip, xshift=-2pt] {2.4 \footnotesize
    SSE};
    \fill[parallel5Color] (69mm, 28mm) rectangle (72mm, 25mm)
    node[labelColor, anchor=mid west, yshift=0.25\baselineskip, xshift=-2pt] {2.4 \footnotesize
    AVX};
    \fill[parallel6Color] (87mm, 28mm) rectangle (90mm, 25mm)
    node[labelColor, anchor=mid west, yshift=0.25\baselineskip, xshift=-2pt] {2.6};
    %
    \fill[parallel7Color] (4mm, 22mm) rectangle (7mm, 19mm) node[labelColor,
    anchor=mid west, yshift=0.25\baselineskip, xshift=-2pt] {2.8};
    \fill[parallel8Color] (15mm, 22mm) rectangle (18mm, 19mm)
    node[labelColor, anchor=mid west, yshift=0.25\baselineskip, xshift=-2pt] {2.9 \footnotesize
    SSE};
    \fill[parallel9Color] (33mm, 22mm) rectangle (36mm, 19mm)
    node[labelColor, anchor=mid west, yshift=0.25\baselineskip, xshift=-2pt] {2.9 \footnotesize
    AVX};
    \fill[parallel10Color] (51mm, 22mm) rectangle (54mm, 19mm)
    node[labelColor, anchor=mid west, yshift=0.25\baselineskip, xshift=-2pt] {2.11 \footnotesize
    SSE};
    \fill[parallel11Color] (69mm, 22mm) rectangle (72mm, 19mm)
    node[labelColor, anchor=mid west, yshift=0.25\baselineskip, xshift=-2pt] {2.11 \footnotesize
    AVX};
\end{tikzpicture}
%
\setlength{\abovecaptionskip}{0pt}%
\end{subfigure}%
\setlength{\abovecaptionskip}{4pt}%
\caption*{Speed-ups varying dataset and data samples layout}
\end{figure}

\vspace{-14pt}
\hypertarget{how-to-use-the-library}{
\section{How to use the library}
\label{how-to-use-the-library}}

%    TODO: Requires C++ 20
%TODO: the data samples can have whatever dimension you want, provided that all the samples have
%the same dimension.
%    TODO: Missing specification of which compiler flags we have used (at least -O3 and
%   -march=native)
%TODO: Cache sizes?
%   TODO: Structure used in the tests to hold pi and lambda --> std::vector

We organized all the sequential and parallel implementations of the clustering algorithm into two
different classes, namely the \texttt{SequentialClustering} and \texttt{ParallelClustering} classes.

% TODO
All these implementations are able to process data samples with any number of attributes,
provided that all the samples have the same number of attributes.
%%%%

The \texttt{main-sample.cpp} file in the \texttt{test/src} directory contains an example
illustrating how to use the library.

\hypertarget{parallel-clustering}{%
\subsection{\texttt{ParallelClustering} class}
\label{parallel-clustering}}
This class, defined in \texttt{ParallelClustering.h} header file, offers several utility methods and
constants allowing the programmer to easily deal with aligned memory. In particular, the
\texttt{computeSseDimension} and \texttt{computeAvxDimension} static methods allow to compute the
number of \texttt{double} elements each data sample must be composed of, taking into account also
possible paddings.

However, the \texttt{cluster} static method is the most important one offered by this class. It
clusters
the given data samples using the specified distance computation algorithm (see
\ref{distance-computation-algorithms} for the list of available algorithms), which
is specified as a template argument so to compile only the code needed to execute the requested
algorithm.
The method allows also to specify the number of threads to use in each of the steps that can be
parallelized, namely:
\begin{enumerate}
\item The computation of the distance between two data samples;
% TODO:
\item The operations needed to fix the structure of the dendrogram after a new point is
added, i.e., fixing the representative of all the points whose representative joined a
cluster containing the new data sample (the step \textit{4} of the algorithm);
%%%%
\item The computation of the square roots of the distances stored in $\lambda$.
\end{enumerate}
Note that the parallelization of such steps can be enabled or disabled at compile time by using the
template parameters of the \texttt{ParallelClustering} class.

\hypertarget{par-data-samples-layout}{
\subsubsection{Data samples memory layout}
\label{par-data-samples-layout}}

The \texttt{cluster} static method allows a certain flexibility in the memory layout of the data
samples. In particular, the method accepts two main organizations:
\begin{enumerate}
\item A contiguous region of memory, where the data samples are stored one after the other.
The method accepts either an iterator iterating over that region, or directly the data
structure holding it. In this case, the \texttt{cluster} method takes care of extracting the
iterator by calling the \texttt{begin} method or, if present, the \texttt{cbegin} one;

\item A data layout organized in two levels.

The first level is merely a container that holds the data samples to cluster (or the iterators
iterating over them). This container is required to be randomly accessible, but it is not
required to be contiguous in memory.

Each element of the second level is a data structure (or an iterator iterating over it) holding a
contiguous region of memory that stores the attributes of one data sample.

For both levels, the \texttt{cluster} method accepts either the iterator over the data structure, or
the data structure itself. In this last case, the method takes care of calling the
\texttt{begin} method, or the \texttt{cbegin} one if present, to obtain an iterator iterating
over the data structure.

\end{enumerate}

\hypertarget{par-pi-lambda-layout}{
\subsubsection{\texorpdfstring{\boldmath$\pi$ and \boldmath$\lambda$}{pi and lambda} memory layouts}
\label{par-pi-lambda-layout}}

The \texttt{cluster} static method allows also a certain flexibility in the memory layout of
$\pi$ and $\lambda$.

In particular, each of them can be any randomly accessible data structure, or an iterator
iterating over it.
In the former case, the method takes care of calling \texttt{begin} method, or the
\texttt{cbegin} one if present, to retrieve the random access iterator iterating over the
specified data structure.

The \texttt{cluster} method assumes the data structures holding $\pi$ and
$\lambda$ to be as big as the number of data samples to cluster.

\hypertarget{distance-computation-algorithms}{
\subsubsection{Distance computation algorithms}
\label{distance-computation-algorithms}}

The \texttt{cluster} stating method allows the programmer to use different algorithms to compute
the distance between two data samples. In particular:
\begin{itemize}
\item \texttt{CLASSICAL}, which computes the Euclidean distances without using any SIMD
instruction;
\item \texttt{SSE} and \texttt{AVX}, which compute the distances using SSE and AVX
instructions respectively;
\item \texttt{SSE\textunderscore OPTIMIZED} and \texttt{AVX\textunderscore OPTIMIZED}, which
compute the distances using
SSE and AVX instructions respectively, and they keep the partial sums in the registers
instead of storing them into a variable (see
\ref{micro-optimization-partial-sum});
\item \texttt{SSE\textunderscore
OPTIMIZED\textunderscore NO\textunderscore SQUARE\textunderscore ROOT} and\\
\texttt{AVX\textunderscore OPTIMIZED\textunderscore NO\textunderscore SQUARE\textunderscore
ROOT}, which act like\\ \texttt{SSE\textunderscore OPTIMIZED} and
\texttt{AVX\textunderscore OPTIMIZED} respectively, but they compute the square of the distances,
without applying the square root operation (see \ref{micro-optimization-no-square-root}).
\end{itemize}

All these constants are defined in the \texttt{DistanceComputers} enumeration, defined in the
\texttt{DistanceComputers.h} header file.

\hypertarget{sequential-clustering}{
\subsection{\texttt{SequentialClustering} class}
\label{sequential-clustering}}

This class, defined in the \texttt{SequentialClustering.h} header file, provides only one method,
the \texttt{cluster} static method, which executes the sequential implementation of the
clustering algorithm on the specified data samples.

\hypertarget{seq-data-samples-layout}{
\subsubsection{Data samples memory layout}
\label{seq-data-samples-layout}}

The \texttt{cluster} static method supports the same data layouts as the \texttt{cluster} method
of the parallel implementation (see \ref{par-data-samples-layout}).

In addition, if the specified data structure (or iterator) is organized in two levels, then this
method does require the first level to be randomly accessible, but it requires only the first
level to be input iterable (or to be an input iterator).

\hypertarget{seq-pi-lambda-layout}{
\subsubsection{\texorpdfstring{\boldmath$\pi$ and \boldmath$\lambda$}{pi and lambda} memory layouts}
\label{seq-pi-lambda-layout}}

The \texttt{cluster} static method supports exactly the same data layouts for both $\pi$
and $\lambda$ as the \texttt{cluster} method
of the parallel implementation (see \ref{par-pi-lambda-layout}).

\hypertarget{timers}{
\subsection{\texttt{Timer} class}
\label{timer}}

If you define the \texttt{TIMERS} macro, both the sequential and parallel implementations of the
clustering algorithm will measure the time taken to execute each step of the algorithm using a
timer.

In particular, each step is measured by a different timer with a specific identifier, where
\texttt{0} measures the time taken to execute all the service operations, \texttt{1} to
\texttt{4} measure the time taken to execute the corresponding steps of the pseudo-code,
and \texttt{5} measures the time taken to compute the square roots of the values in
$\lambda$. This time is measured only for the parallel implementation if the distance
computation algorithm is one of \texttt{SSE\textunderscore
OPTIMIZED\textunderscore NO\textunderscore SQUARE\textunderscore ROOT} or\\
\texttt{AVX\textunderscore OPTIMIZED\textunderscore NO\textunderscore SQUARE\textunderscore ROOT}.

The \texttt{Timer}, defined in the \texttt{Timer.h} header file, offers the \texttt{print} and
\texttt{printTotal} static methods to print to the console the times measured by each timer.

\end{document}
